%%%%%%%%%%%%%%%%%%%%%%%%%%%%%%%%%%%%%%%%%
% Beamer Presentation
% LaTeX Template
% Version 1.0 (10/11/12)
%
% This template has been downloaded from:
% http://www.LaTeXTemplates.com
%
% License:
% CC BY-NC-SA 3.0 (http://creativecommons.org/licenses/by-nc-sa/3.0/)
%
%%%%%%%%%%%%%%%%%%%%%%%%%%%%%%%%%%%%%%%%%

%----------------------------------------------------------------------------------------
%	PACKAGES AND THEMES
%----------------------------------------------------------------------------------------

\documentclass{beamer}

\mode<presentation> {

% The Beamer class comes with a number of default slide themes
% which change the colors and layouts of slides. Below this is a list
% of all the themes, uncomment each in turn to see what they look like.

%\usetheme{default}
%\usetheme{AnnArbor}
%\usetheme{Antibes}
%\usetheme{Bergen}
%\usetheme{Berkeley}
%\usetheme{Berlin}
%\usetheme{Boadilla}
%\usetheme{CambridgeUS}
%\usetheme{Copenhagen}
%\usetheme{Darmstadt}
%\usetheme{Dresden}
%\usetheme{Frankfurt}
%\usetheme{Goettingen}
%\usetheme{Hannover}
%\usetheme{Ilmenau}
%\usetheme{JuanLesPins}
%\usetheme{Luebeck}
\usetheme{Madrid}
%\usetheme{Malmoe}
%\usetheme{Marburg}
%\usetheme{Montpellier}
%\usetheme{PaloAlto}
%\usetheme{Pittsburgh}
%\usetheme{Rochester}
%\usetheme{Singapore}
%\usetheme{Szeged}
%\usetheme{Warsaw}

% As well as themes, the Beamer class has a number of color themes
% for any slide theme. Uncomment each of these in turn to see how it
% changes the colors of your current slide theme.

%\usecolortheme{albatross}
%\usecolortheme{beaver}
%\usecolortheme{beetle}
%\usecolortheme{crane}
%\usecolortheme{dolphin}
%\usecolortheme{dove}
%\usecolortheme{fly}
%\usecolortheme{lily}
%\usecolortheme{orchid}
%\usecolortheme{rose}
%\usecolortheme{seagull}
%\usecolortheme{seahorse}
%\usecolortheme{whale}
%\usecolortheme{wolverine}

%\setbeamertemplate{footline} % To remove the footer line in all slides uncomment this line
%\setbeamertemplate{footline}[page number] % To replace the footer line in all slides with a simple slide count uncomment this line

%\setbeamertemplate{navigation symbols}{} % To remove the navigation symbols from the bottom of all slides uncomment this line
}

\usepackage{graphicx} % Allows including images
\usepackage{booktabs} % Allows the use of \toprule, \midrule and \bottomrule in tables

%----------------------------------------------------------------------------------------
%	TITLE PAGE
%----------------------------------------------------------------------------------------

\title[]{Processing blood cells image using deep learning techniques } % The short title appears at the bottom of every slide, the full title is only on the title page

\author{Nguyen Tat Hung} % Your name
\institute[USTH] % Your institution as it will appear on the bottom of every slide, may be shorthand to save space
{
University science and technology of Ha Noi \\ % Your institution for the title page
\medskip
\textit{hungnt.m20ict@st.usth.edu.vn} % Your email address
}
\date{\today} % Date, can be changed to a custom date

\begin{document}

\begin{frame}
\titlepage % Print the title page as the first slide
\end{frame}

%------------------------------------------------

\begin{frame}
\frametitle{project's purpose}
Classify blood cells based on microscope images
\begin{itemize}
\item Computer replace human process result from microscope for quick results processing
\item Apply deep learning to process analytic blood cells it help for doctor get result faster with analytic manually
\item Implement an approach automatic detect blood cells base on image capture from microscope
\item In this project we have 8 class blood cell types (Monocytes,Lymphocytes,Basophils,Eosinophils,Neutrophils,Platelets,Leukocytes,Red blood cells)
\end{itemize}
\end{frame}

%------------------------------------------------

\begin{frame}
\frametitle{blood diseases}
About blood diseases we are have four main group diseases
\begin{itemize}
\item Leukemia or blood cancer
\item Malaria 
\item Anemia group
\item Myelodysplastic syndrome
\end{itemize}
\end{frame}

%------------------------------------------------

\begin{frame}
\frametitle{Leukemia}
Blood cancer, also known as leukemia or white blood disease
\begin{itemize}
\item leukemia can be divided into acute or chronic leukemia
\item it can be divided into lymphoid leukemia or myeloid leukemia 
\end{itemize}
\end{frame}
%------------------------------------------------

\begin{frame}
\frametitle{Malaria}
Malaria is an infection caused by parasites of the genus Plasmodium
\end{frame}
%------------------------------------------------

\begin{frame}
\frametitle{Anemia group}
\begin{itemize}
\item it causes sickle cell disease
\item Anemia is a group of inherited anemias caused by a lack of healthy red blood cells to carry oxygen throughout the body
\end{itemize}
\centering
\includegraphics<1>[scale=0.5,width=8cm]{img/img2}
\end{frame}
%------------------------------------------------

\begin{frame}
\frametitle{Myelodysplastic syndrome}
\begin{itemize}
\item Myelodysplastic syndromes (MDS) are a group of disorders associated with a decrease in peripheral blood cells, disorders of hematopoietic progenitors, and increased or decreased bone marrow cells, with a high risk of transforming into disease
\end{itemize}
\end{frame}
%------------------------------------------------


\begin{frame}
\frametitle{Model on a CNN network} % Table of contents slide, comment this block out to remove it
\centering
\includegraphics<1>[scale=0.4,width=10cm]{img/img1}
\end{frame}
%------------------------------------------------


\begin{frame}
\frametitle{About method and algorithm using in this project} % Table of contents slide, comment this block out to remove it
\begin{itemize}
\item YOLOv5s 
\item YOLOv8s
\item SSD300 (VGG16)
\item Faster RCNN (ResNet50)
\end{itemize}
Compare model and choose best algorithm apply for this project when run in environment Linux Ubuntu Version 22.04 VGA RTX 3080 Ti
\centering
\includegraphics<1>[scale=0.4,width=10cm]{img/img6}
\end{frame}

%------------------------------------------------

\begin{frame}
\frametitle{About Dataset BCCD} % Table of contents slide, comment this block out to remove it
\begin{itemize}
\item Source dataset is BCCD dataset get at \url{https://github.com/Shenggan/BCCD_Dataset/tree/master}
\item This is dataset hava VOC Format have Image Type JPEG have size WxH = 640x480
\item Problem this dataset is : A large difference in the number of cells , and data not assign label
\item Solution resolve: Divide data and data balance, and assign label and approved from doctor in bach mai hospital
\item This dataset have 410 images separator 205 images for Train set and 292 Validation set. And have 3 label for each cells
\end{itemize}
\end{frame}

%------------------------------------------------

\begin{frame}
\frametitle{Demo image capture from microscope} % Table of contents slide, comment this block out to remove it
\centering
\includegraphics<1>[scale=0.4,width=9cm]{img/bccd-sample}
\end{frame}
%------------------------------------------------

\begin{frame}
\frametitle{Blood cell when detect by deep learning} % Table of contents slide, comment this block out to remove it
\centering
\includegraphics<1>[scale=0.4,width=9cm]{img/image_0_1}
\end{frame}

%------------------------------------------------
\begin{frame}
\frametitle{The relationship of mAP with epoch} % Table of contents slide, comment this block out to remove it
\centering
\includegraphics<1>[scale=1,width=10cm]{img/map}
\end{frame}

%------------------------------------------------
\begin{frame}
\frametitle{mAP at IoU=50} % Table of contents slide, comment this block out to remove it
\centering
\includegraphics<1>[scale=1,width=10cm]{img/img5}
\end{frame}

%------------------------------------------------

\begin{frame}
\frametitle{Loss}
\begin{itemize}
\item represents the error between the model's prediction and reality, calculated by a loss function
\item Loss is calculated on the training set (training loss) and validation set (validation loss) during the model training process
\item A good model will have training loss and validation loss that gradually decreases and stabilizes during the training process
\item Overfitting phenomenon: training loss gradually decreases but validation loss gradually increases, proving that the model has poor accuracy when encountering new, strange data
\end{itemize}
\end{frame}

%------------------------------------------------
\begin{frame}
\frametitle{The relationship of training loss rate with epoch} % Table of contents slide, comment this block out to remove it
\centering
\includegraphics<1>[scale=1,width=10cm]{img/img3}
\end{frame}

%------------------------------------------------
\begin{frame}
\frametitle{Confusion matrix} % Table of contents slide, comment this block out to remove it
\centerline{The matrix summarizes all the model's predictions}
\centerline{on the test/validation data set}
\centering
\includegraphics<1>[scale=1,width=10cm,height=6cm]{img/img4}
\end{frame}

%------------------------------------------------
\begin{frame}
\frametitle{Conclusion} % Table of contents slide, comment this block out to remove it
\begin{itemize}
\item Dataset are greatly related to the accuracy of the model.
Use a balanced data set for higher accuracy than original data
\item Loss is parameter important in model CNN because it reflect accuracy of model
\item The large difference in the number of cells of all kinds will affect the "learning" ability of the model
\item The optimal model selection for the project. determine the success or failure of the project in the classification of blood cells of the machine model
\end{itemize}
\end{frame}

%------------------------------------------------
\begin{frame}
\frametitle{Future Works} % Table of contents slide, comment this block out to remove it
\begin{itemize}
\item The initial training process is expensive and takes a lot of time. It is difficult to deploy without enough data. so we need to prepare dataset large enough to train machine models
\item During the data analysis, the machine also continuously "learned" to improve the accuracy over time. So the system will increase accuracy after a period of operation
\end{itemize}
\end{frame}

%------------------------------------------------

\begin{frame}
\Huge{\centerline{THANKS FOR}}
\Huge{\centerline{YOUR ATTENTION}}
\end{frame}

%----------------------------------------------------------------------------------------

\end{document} 